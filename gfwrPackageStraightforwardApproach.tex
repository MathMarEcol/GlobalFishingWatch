% Options for packages loaded elsewhere
\PassOptionsToPackage{unicode}{hyperref}
\PassOptionsToPackage{hyphens}{url}
%
\documentclass[
]{article}
\usepackage{amsmath,amssymb}
\usepackage{iftex}
\ifPDFTeX
  \usepackage[T1]{fontenc}
  \usepackage[utf8]{inputenc}
  \usepackage{textcomp} % provide euro and other symbols
\else % if luatex or xetex
  \usepackage{unicode-math} % this also loads fontspec
  \defaultfontfeatures{Scale=MatchLowercase}
  \defaultfontfeatures[\rmfamily]{Ligatures=TeX,Scale=1}
\fi
\usepackage{lmodern}
\ifPDFTeX\else
  % xetex/luatex font selection
\fi
% Use upquote if available, for straight quotes in verbatim environments
\IfFileExists{upquote.sty}{\usepackage{upquote}}{}
\IfFileExists{microtype.sty}{% use microtype if available
  \usepackage[]{microtype}
  \UseMicrotypeSet[protrusion]{basicmath} % disable protrusion for tt fonts
}{}
\makeatletter
\@ifundefined{KOMAClassName}{% if non-KOMA class
  \IfFileExists{parskip.sty}{%
    \usepackage{parskip}
  }{% else
    \setlength{\parindent}{0pt}
    \setlength{\parskip}{6pt plus 2pt minus 1pt}}
}{% if KOMA class
  \KOMAoptions{parskip=half}}
\makeatother
\usepackage{xcolor}
\usepackage[margin=1in]{geometry}
\usepackage{color}
\usepackage{fancyvrb}
\newcommand{\VerbBar}{|}
\newcommand{\VERB}{\Verb[commandchars=\\\{\}]}
\DefineVerbatimEnvironment{Highlighting}{Verbatim}{commandchars=\\\{\}}
% Add ',fontsize=\small' for more characters per line
\usepackage{framed}
\definecolor{shadecolor}{RGB}{248,248,248}
\newenvironment{Shaded}{\begin{snugshade}}{\end{snugshade}}
\newcommand{\AlertTok}[1]{\textcolor[rgb]{0.94,0.16,0.16}{#1}}
\newcommand{\AnnotationTok}[1]{\textcolor[rgb]{0.56,0.35,0.01}{\textbf{\textit{#1}}}}
\newcommand{\AttributeTok}[1]{\textcolor[rgb]{0.13,0.29,0.53}{#1}}
\newcommand{\BaseNTok}[1]{\textcolor[rgb]{0.00,0.00,0.81}{#1}}
\newcommand{\BuiltInTok}[1]{#1}
\newcommand{\CharTok}[1]{\textcolor[rgb]{0.31,0.60,0.02}{#1}}
\newcommand{\CommentTok}[1]{\textcolor[rgb]{0.56,0.35,0.01}{\textit{#1}}}
\newcommand{\CommentVarTok}[1]{\textcolor[rgb]{0.56,0.35,0.01}{\textbf{\textit{#1}}}}
\newcommand{\ConstantTok}[1]{\textcolor[rgb]{0.56,0.35,0.01}{#1}}
\newcommand{\ControlFlowTok}[1]{\textcolor[rgb]{0.13,0.29,0.53}{\textbf{#1}}}
\newcommand{\DataTypeTok}[1]{\textcolor[rgb]{0.13,0.29,0.53}{#1}}
\newcommand{\DecValTok}[1]{\textcolor[rgb]{0.00,0.00,0.81}{#1}}
\newcommand{\DocumentationTok}[1]{\textcolor[rgb]{0.56,0.35,0.01}{\textbf{\textit{#1}}}}
\newcommand{\ErrorTok}[1]{\textcolor[rgb]{0.64,0.00,0.00}{\textbf{#1}}}
\newcommand{\ExtensionTok}[1]{#1}
\newcommand{\FloatTok}[1]{\textcolor[rgb]{0.00,0.00,0.81}{#1}}
\newcommand{\FunctionTok}[1]{\textcolor[rgb]{0.13,0.29,0.53}{\textbf{#1}}}
\newcommand{\ImportTok}[1]{#1}
\newcommand{\InformationTok}[1]{\textcolor[rgb]{0.56,0.35,0.01}{\textbf{\textit{#1}}}}
\newcommand{\KeywordTok}[1]{\textcolor[rgb]{0.13,0.29,0.53}{\textbf{#1}}}
\newcommand{\NormalTok}[1]{#1}
\newcommand{\OperatorTok}[1]{\textcolor[rgb]{0.81,0.36,0.00}{\textbf{#1}}}
\newcommand{\OtherTok}[1]{\textcolor[rgb]{0.56,0.35,0.01}{#1}}
\newcommand{\PreprocessorTok}[1]{\textcolor[rgb]{0.56,0.35,0.01}{\textit{#1}}}
\newcommand{\RegionMarkerTok}[1]{#1}
\newcommand{\SpecialCharTok}[1]{\textcolor[rgb]{0.81,0.36,0.00}{\textbf{#1}}}
\newcommand{\SpecialStringTok}[1]{\textcolor[rgb]{0.31,0.60,0.02}{#1}}
\newcommand{\StringTok}[1]{\textcolor[rgb]{0.31,0.60,0.02}{#1}}
\newcommand{\VariableTok}[1]{\textcolor[rgb]{0.00,0.00,0.00}{#1}}
\newcommand{\VerbatimStringTok}[1]{\textcolor[rgb]{0.31,0.60,0.02}{#1}}
\newcommand{\WarningTok}[1]{\textcolor[rgb]{0.56,0.35,0.01}{\textbf{\textit{#1}}}}
\usepackage{graphicx}
\makeatletter
\def\maxwidth{\ifdim\Gin@nat@width>\linewidth\linewidth\else\Gin@nat@width\fi}
\def\maxheight{\ifdim\Gin@nat@height>\textheight\textheight\else\Gin@nat@height\fi}
\makeatother
% Scale images if necessary, so that they will not overflow the page
% margins by default, and it is still possible to overwrite the defaults
% using explicit options in \includegraphics[width, height, ...]{}
\setkeys{Gin}{width=\maxwidth,height=\maxheight,keepaspectratio}
% Set default figure placement to htbp
\makeatletter
\def\fps@figure{htbp}
\makeatother
\setlength{\emergencystretch}{3em} % prevent overfull lines
\providecommand{\tightlist}{%
  \setlength{\itemsep}{0pt}\setlength{\parskip}{0pt}}
\setcounter{secnumdepth}{-\maxdimen} % remove section numbering
\ifLuaTeX
  \usepackage{selnolig}  % disable illegal ligatures
\fi
\IfFileExists{bookmark.sty}{\usepackage{bookmark}}{\usepackage{hyperref}}
\IfFileExists{xurl.sty}{\usepackage{xurl}}{} % add URL line breaks if available
\urlstyle{same}
\hypersetup{
  pdftitle={gfwr package tool},
  pdfauthor={Kilian GRIHAULT BARREIRO},
  hidelinks,
  pdfcreator={LaTeX via pandoc}}

\title{gfwr package tool}
\author{Kilian GRIHAULT BARREIRO}
\date{2024-01-14}

\begin{document}
\maketitle

\hypertarget{how-to-use-global-fishing-watch-r-package-gfwr}{%
\section{\texorpdfstring{How to use : Global Fishing Watch R Package
(\texttt{gfwr})}{How to use : Global Fishing Watch R Package (gfwr)}}\label{how-to-use-global-fishing-watch-r-package-gfwr}}

The \texttt{gfwr} package provides convenient functions to pull GFW data
directly into R into usable formats.\\
It contains three main functions, including
\texttt{get\_vessel\_info()}, \texttt{get\_event()} and
\texttt{get\_raster()}. The two first being devoted to retrieving
information and features on one ore several specific vessels. The last
is of particular interest to us because it allows us to gather
information from global fishing watch raster on the fishing effort
(further details in the function appropriate section).

The time spent fishing is computed using Automatic Identification System
(AIS) data, which is transmitted by most industrial fishing vessels. The
AIS data provides information on the location, speed, and direction of
the vessel, which can be used to identify when the vessel is actively
fishing.

\hypertarget{ais-caveats-and-limitations}{%
\paragraph{AIS Caveats and
limitations}\label{ais-caveats-and-limitations}}

The AIS coverage of vessels has several limitations such as : 1. The
number of vessels that are captured (AIS provides approximately 70'000
of the 2.8 million identified fishing vessels). 2. The size of the
vessels (52-85\% for vessels larger than 24 meters against 1\% for
vessels under 12 meters). \emph{Good to know: IMO mandates AIS for most
vessels larger than 36 meters.} 3. AIS interference with each other in
areas of high vessel density. 4. Some terrestrial satellites only
receive messages near shore.

\hypertarget{installation}{%
\subsection{Installation}\label{installation}}

\begin{Shaded}
\begin{Highlighting}[]
\NormalTok{remotes}\SpecialCharTok{::}\FunctionTok{install\_github}\NormalTok{(}\StringTok{"GlobalFishingWatch/gfwr"}\NormalTok{)}
\end{Highlighting}
\end{Shaded}

\begin{Shaded}
\begin{Highlighting}[]
\FunctionTok{library}\NormalTok{(gfwr)}
\FunctionTok{library}\NormalTok{(ggplot2)}
\FunctionTok{library}\NormalTok{(tidyverse)}
\FunctionTok{library}\NormalTok{(sf)}
\end{Highlighting}
\end{Shaded}

\hypertarget{api}{%
\subsection{API}\label{api}}

\begin{description}
\item[To access GFW APIs, you need to : 1. register for a GFW account
here]
\textless{}\href{https://tinyurl.com/2pr5nxxk}{Shortened
URL}\textgreater{} 2. Request API key here :
\url{https://globalfishingwatch.org/our-apis/tokens}
\end{description}

Once you have your token, add it to your .Renviron file (by executing
the chunk below), by writing (GFW\_TOKEN = ``YOUR\_TOKEN'') in the file.
\emph{(You could be asked to restart R for changes to take effect.)}

\begin{Shaded}
\begin{Highlighting}[]
\NormalTok{usethis}\SpecialCharTok{::}\FunctionTok{edit\_r\_environ}\NormalTok{()}
\end{Highlighting}
\end{Shaded}

We save the key in an object that will be used in gfwr functions.

\begin{Shaded}
\begin{Highlighting}[]
\NormalTok{key }\OtherTok{\textless{}{-}} \FunctionTok{gfw\_auth}\NormalTok{()}
\end{Highlighting}
\end{Shaded}

\hypertarget{fishing-effort-visualization}{%
\subsection{Fishing effort
visualization}\label{fishing-effort-visualization}}

A region\_id is necessary to use the \texttt{get\_raster} function.

\begin{Shaded}
\begin{Highlighting}[]
\NormalTok{region\_id }\OtherTok{\textless{}{-}} \FunctionTok{get\_region\_id}\NormalTok{(}
  \AttributeTok{region\_name =} \StringTok{"Australia"}\NormalTok{,}
  \AttributeTok{region\_source =} \StringTok{"eez"}\NormalTok{,}
  \AttributeTok{key =}\NormalTok{ key}
\NormalTok{)}\SpecialCharTok{$}\NormalTok{id}
\end{Highlighting}
\end{Shaded}

The \texttt{get\_raster} function gets a raster of fishing effort from
the API and converts the response to a data frame which contains
occurrences for each vessel and for each grid cell (data is binned into
grid cells of different resolution), the \texttt{Vessel\ IDs},
\texttt{Flag}, \texttt{Geartype} and \texttt{Apparent\ fishing\ Hours}
which are basically the amount of fishing hours of each vessel per grid
cell (\texttt{geometry}).

Data can be provided through : - \texttt{daily}, \texttt{monthly} and
\texttt{yearly} temporal resolutions. - \texttt{low} (0.1 deg) and
\texttt{high} (0.01 deg) spatial resolutions. - \texttt{vessel\_id},
\texttt{flag}, \texttt{gearType}, \texttt{flagAndGearType}.

\begin{Shaded}
\begin{Highlighting}[]
\NormalTok{data }\OtherTok{\textless{}{-}} \FunctionTok{get\_raster}\NormalTok{(}
  \AttributeTok{spatial\_resolution =} \StringTok{"low"}\NormalTok{,}
  \AttributeTok{temporal\_resolution =} \StringTok{"monthly"}\NormalTok{,}
  \AttributeTok{group\_by =} \StringTok{"flagAndGearType"}\NormalTok{,}
  \AttributeTok{date\_range =} \StringTok{"2022{-}01{-}01,2023{-}01{-}01"}\NormalTok{,}
  \AttributeTok{region =}\NormalTok{ region\_id,}
  \AttributeTok{region\_source =} \StringTok{"eez"}\NormalTok{,}
  \AttributeTok{key =}\NormalTok{ key}
\NormalTok{)}
\end{Highlighting}
\end{Shaded}

\emph{(You can remove the option} \texttt{message\ =\ FALSE} \emph{to
see the columns types.)}

\hypertarget{get_raster-caveats-and-limitations.}{%
\paragraph{\texorpdfstring{\texttt{get\_raster} caveats and
limitations.}{get\_raster caveats and limitations.}}\label{get_raster-caveats-and-limitations.}}

Date range is limited to 1-year. Nevertheless, with some modifications,
we can get round these problems through \texttt{get\_gfwFishingEffort}.

\hypertarget{get_gfwfishingeffort-function.}{%
\paragraph{\texorpdfstring{\texttt{get\_gfwFishingEffort}
function.}{get\_gfwFishingEffort function.}}\label{get_gfwfishingeffort-function.}}

The \texttt{get\_gfwFishingEffort} function recover the data of Global
Fishing Watch and returns it as a sf object. We have the same parameters
than the \texttt{get\_raster} function, plus \texttt{n\_crs} which is
the crs for the sf\_modification Different possible values can be
combined and are : - \texttt{Time\ Range}, \texttt{Flag},
\texttt{Geartype}. \emph{(A combination can be : c(`Time
Range',`Geartype'), if you want to get the sum of fishing hours per date
and geartype, for example you want to display the drifting longline
fishing in a specific year)} \textbf{Notes :} 1. For the moment we are
limited to the EEZs of each region, but we can potentially restrict the
working area to specific MPAs (further details in the gfwr package). 2.
Days indicated in the\_\_ \texttt{start\_date} \textbf{and}
\texttt{end\_date} \_\_variables are included in the data recovery.

\begin{Shaded}
\begin{Highlighting}[]
\NormalTok{get\_gfwData }\OtherTok{\textless{}{-}} \ControlFlowTok{function}\NormalTok{(region, start\_date, end\_date, temp\_res,}
                        \AttributeTok{spat\_res =} \StringTok{"low"}\NormalTok{,}
                        \AttributeTok{key =}\NormalTok{ gfwr}\SpecialCharTok{::}\FunctionTok{gfw\_auth}\NormalTok{(),}
                        \AttributeTok{cCRS =} \StringTok{"EPSG:4326"}\NormalTok{,}
                        \AttributeTok{compress =} \ConstantTok{FALSE}\NormalTok{) \{}
\NormalTok{  region\_id }\OtherTok{\textless{}{-}}\NormalTok{ gfwr}\SpecialCharTok{::}\FunctionTok{get\_region\_id}\NormalTok{(}
    \AttributeTok{region\_name =}\NormalTok{ region,}
    \AttributeTok{region\_source =} \StringTok{"eez"}\NormalTok{,}
    \AttributeTok{key =}\NormalTok{ key}
\NormalTok{  )}\SpecialCharTok{$}\NormalTok{id[}\DecValTok{1}\NormalTok{]}

  \CommentTok{\# Convert dates into Date objects}
\NormalTok{  start\_date }\OtherTok{\textless{}{-}} \FunctionTok{as.Date}\NormalTok{(start\_date, }\AttributeTok{format =} \StringTok{"\%Y{-}\%m{-}\%d"}\NormalTok{)}
\NormalTok{  end\_date }\OtherTok{\textless{}{-}} \FunctionTok{as.Date}\NormalTok{(end\_date, }\AttributeTok{format =} \StringTok{"\%Y{-}\%m{-}\%d"}\NormalTok{)}

  \CommentTok{\# Function to obtain data for a specific date range}
\NormalTok{  get\_data\_for\_range }\OtherTok{\textless{}{-}} \ControlFlowTok{function}\NormalTok{(start\_date, end\_date) \{}
\NormalTok{    date\_range }\OtherTok{\textless{}{-}} \FunctionTok{paste}\NormalTok{(start\_date, end\_date, }\AttributeTok{sep =} \StringTok{","}\NormalTok{)}

\NormalTok{    data }\OtherTok{\textless{}{-}}\NormalTok{ gfwr}\SpecialCharTok{::}\FunctionTok{get\_raster}\NormalTok{(}
      \AttributeTok{spatial\_resolution =}\NormalTok{ spat\_res,}
      \AttributeTok{temporal\_resolution =}\NormalTok{ temp\_res,}
      \AttributeTok{group\_by =} \StringTok{"flagAndGearType"}\NormalTok{,}
      \AttributeTok{date\_range =}\NormalTok{ date\_range,}
      \AttributeTok{region =}\NormalTok{ region\_id,}
      \AttributeTok{region\_source =} \StringTok{"eez"}\NormalTok{,}
      \AttributeTok{key =}\NormalTok{ key}
\NormalTok{    )}

    \FunctionTok{return}\NormalTok{(data)}
\NormalTok{  \}}

  \CommentTok{\# Check whether the date range is less than or equal to 366 days}
  \ControlFlowTok{if}\NormalTok{ (}\FunctionTok{as.numeric}\NormalTok{(}\FunctionTok{difftime}\NormalTok{(end\_date, start\_date, }\AttributeTok{units =} \StringTok{"days"}\NormalTok{)) }\SpecialCharTok{\textless{}=} \DecValTok{366}\NormalTok{) \{}
    \CommentTok{\# If yes, obtain data for the entire date range}
\NormalTok{    data\_df }\OtherTok{\textless{}{-}} \FunctionTok{get\_data\_for\_range}\NormalTok{(start\_date, end\_date)}
\NormalTok{  \} }\ControlFlowTok{else}\NormalTok{ \{}
    \CommentTok{\# If not, divide the date range into 366{-}day chunks and obtain the data for each chunk.}
\NormalTok{    date\_chunks }\OtherTok{\textless{}{-}} \FunctionTok{seq}\NormalTok{(start\_date, end\_date, }\AttributeTok{by =} \StringTok{"366 days"}\NormalTok{)}
\NormalTok{    data\_df }\OtherTok{\textless{}{-}}\NormalTok{ purrr}\SpecialCharTok{::}\FunctionTok{map\_dfr}\NormalTok{(}
\NormalTok{      date\_chunks,}
      \SpecialCharTok{\textasciitilde{}} \FunctionTok{get\_data\_for\_range}\NormalTok{(.x, }\FunctionTok{min}\NormalTok{(.x }\SpecialCharTok{+} \DecValTok{365}\NormalTok{, end\_date))}
\NormalTok{    )}
\NormalTok{  \}}

  \ControlFlowTok{if}\NormalTok{ (}\FunctionTok{isTRUE}\NormalTok{(compress)) \{}
    \CommentTok{\# GFW data will always be "EPSG:4326". No need to have CRS as an option here}

\NormalTok{    data\_df }\OtherTok{\textless{}{-}}\NormalTok{ data\_df }\SpecialCharTok{\%\textgreater{}\%}
\NormalTok{      dplyr}\SpecialCharTok{::}\FunctionTok{select}\NormalTok{(}\StringTok{"Lon"}\NormalTok{, }\StringTok{"Lat"}\NormalTok{, }\StringTok{"Apparent Fishing Hours"}\NormalTok{) }\SpecialCharTok{\%\textgreater{}\%}
\NormalTok{      dplyr}\SpecialCharTok{::}\FunctionTok{group\_by}\NormalTok{(Lon, Lat) }\SpecialCharTok{\%\textgreater{}\%}
\NormalTok{      dplyr}\SpecialCharTok{::}\FunctionTok{summarise}\NormalTok{(}\StringTok{"Apparent Fishing Hours"} \OtherTok{=} \FunctionTok{sum}\NormalTok{(}\StringTok{\textasciigrave{}}\AttributeTok{Apparent Fishing Hours}\StringTok{\textasciigrave{}}\NormalTok{,}
        \AttributeTok{na.rm =} \ConstantTok{TRUE}
\NormalTok{      )) }\SpecialCharTok{\%\textgreater{}\%}
\NormalTok{      dplyr}\SpecialCharTok{::}\FunctionTok{ungroup}\NormalTok{()}

\NormalTok{    data\_sf }\OtherTok{\textless{}{-}}\NormalTok{ data\_df }\SpecialCharTok{\%\textgreater{}\%}
\NormalTok{      terra}\SpecialCharTok{::}\FunctionTok{rast}\NormalTok{(}\AttributeTok{type =} \StringTok{"xyz"}\NormalTok{, }\AttributeTok{crs =} \StringTok{"EPSG:4326"}\NormalTok{) }\SpecialCharTok{\%\textgreater{}\%} \CommentTok{\# Convert to polygons for easier use}
\NormalTok{      terra}\SpecialCharTok{::}\FunctionTok{as.polygons}\NormalTok{(}
        \AttributeTok{trunc =} \ConstantTok{FALSE}\NormalTok{,}
        \AttributeTok{dissolve =} \ConstantTok{FALSE}\NormalTok{,}
        \AttributeTok{na.rm =} \ConstantTok{TRUE}\NormalTok{,}
        \AttributeTok{round =} \ConstantTok{FALSE}
\NormalTok{      ) }\SpecialCharTok{\%\textgreater{}\%}
\NormalTok{      sf}\SpecialCharTok{::}\FunctionTok{st\_as\_sf}\NormalTok{()}

    \ControlFlowTok{if}\NormalTok{ (}\FunctionTok{dim}\NormalTok{(data\_df)[}\DecValTok{1}\NormalTok{] }\SpecialCharTok{!=} \FunctionTok{dim}\NormalTok{(data\_sf)[}\DecValTok{1}\NormalTok{]) \{}
      \FunctionTok{stop}\NormalTok{(}\StringTok{"Data dimensions of df and sf data do not match after conversion"}\NormalTok{)}
\NormalTok{    \}}
\NormalTok{  \} }\ControlFlowTok{else} \ControlFlowTok{if}\NormalTok{ (}\FunctionTok{isFALSE}\NormalTok{(compress)) \{}
    \CommentTok{\# Combine data frames in the list into one data frame}
\NormalTok{    data\_df }\OtherTok{\textless{}{-}} \FunctionTok{bind\_rows}\NormalTok{(data\_df)}

    \CommentTok{\# Separate the "Time Range" column based on the specified temp\_res}
    \ControlFlowTok{if}\NormalTok{ (temp\_res }\SpecialCharTok{==} \StringTok{"yearly"}\NormalTok{) \{}
\NormalTok{      data\_sf }\OtherTok{\textless{}{-}}\NormalTok{ data\_df }\SpecialCharTok{\%\textgreater{}\%}
\NormalTok{        dplyr}\SpecialCharTok{::}\FunctionTok{mutate}\NormalTok{(}\AttributeTok{Year =} \StringTok{\textasciigrave{}}\AttributeTok{Time Range}\StringTok{\textasciigrave{}}\NormalTok{) }\SpecialCharTok{\%\textgreater{}\%}
\NormalTok{        sf}\SpecialCharTok{::}\FunctionTok{st\_as\_sf}\NormalTok{(}
          \AttributeTok{coords =} \FunctionTok{c}\NormalTok{(}\StringTok{"Lon"}\NormalTok{, }\StringTok{"Lat"}\NormalTok{),}
          \AttributeTok{crs =} \StringTok{"EPSG:4326"}
\NormalTok{        )}
\NormalTok{    \} }\ControlFlowTok{else}\NormalTok{ \{}
      \CommentTok{\# Sinon, séparer la colonne "Time Range" selon le temp\_res spécifié}
      \ControlFlowTok{if}\NormalTok{ (temp\_res }\SpecialCharTok{==} \StringTok{"monthly"}\NormalTok{) \{}
\NormalTok{        data\_sf }\OtherTok{\textless{}{-}}\NormalTok{ data\_df }\SpecialCharTok{\%\textgreater{}\%}
\NormalTok{          tidyr}\SpecialCharTok{::}\FunctionTok{separate}\NormalTok{(}\StringTok{"Time Range"}\NormalTok{,}
            \AttributeTok{into =} \FunctionTok{c}\NormalTok{(}\StringTok{"Year"}\NormalTok{, }\StringTok{"Month"}\NormalTok{),}
            \AttributeTok{sep =} \StringTok{"{-}"}\NormalTok{,}
            \AttributeTok{remove =} \ConstantTok{FALSE}
\NormalTok{          ) }\SpecialCharTok{\%\textgreater{}\%}
\NormalTok{          sf}\SpecialCharTok{::}\FunctionTok{st\_as\_sf}\NormalTok{(}
            \AttributeTok{coords =} \FunctionTok{c}\NormalTok{(}\StringTok{"Lon"}\NormalTok{, }\StringTok{"Lat"}\NormalTok{),}
            \AttributeTok{crs =} \StringTok{"EPSG:4326"}
\NormalTok{          )}
\NormalTok{      \} }\ControlFlowTok{else} \ControlFlowTok{if}\NormalTok{ (temp\_res }\SpecialCharTok{==} \StringTok{"daily"}\NormalTok{) \{}
\NormalTok{        data\_sf }\OtherTok{\textless{}{-}}\NormalTok{ data\_df }\SpecialCharTok{\%\textgreater{}\%}
\NormalTok{          tidyr}\SpecialCharTok{::}\FunctionTok{separate}\NormalTok{(}\StringTok{"Time Range"}\NormalTok{,}
            \AttributeTok{into =} \FunctionTok{c}\NormalTok{(}\StringTok{"Year"}\NormalTok{, }\StringTok{"Month"}\NormalTok{, }\StringTok{"Day"}\NormalTok{),}
            \AttributeTok{sep =} \StringTok{"{-}"}\NormalTok{,}
            \AttributeTok{remove =} \ConstantTok{FALSE}
\NormalTok{          ) }\SpecialCharTok{\%\textgreater{}\%}
\NormalTok{          sf}\SpecialCharTok{::}\FunctionTok{st\_as\_sf}\NormalTok{(}
            \AttributeTok{coords =} \FunctionTok{c}\NormalTok{(}\StringTok{"Lon"}\NormalTok{, }\StringTok{"Lat"}\NormalTok{),}
            \AttributeTok{crs =} \StringTok{"EPSG:4326"}
\NormalTok{          )}
\NormalTok{      \}}
\NormalTok{    \}}
\NormalTok{  \}}

  \CommentTok{\# But you may wish to return the data in a different CRS. For this you need transform}
  \ControlFlowTok{if}\NormalTok{ (}\FunctionTok{isFALSE}\NormalTok{(cCRS }\SpecialCharTok{==} \StringTok{"EPSG:4326"}\NormalTok{)) \{}
\NormalTok{    data\_sf }\OtherTok{\textless{}{-}}\NormalTok{ data\_sf }\SpecialCharTok{\%\textgreater{}\%}
\NormalTok{      sf}\SpecialCharTok{::}\FunctionTok{st\_transform}\NormalTok{(}\AttributeTok{crs =}\NormalTok{ cCRS)}
\NormalTok{  \}}

  \FunctionTok{return}\NormalTok{(data\_sf)}
\NormalTok{\}}
\end{Highlighting}
\end{Shaded}

To get the raw values given by GFW we do not need to specify any
variable of gathering in \texttt{group\_by\_vars} parameter.

\begin{Shaded}
\begin{Highlighting}[]
\NormalTok{data\_sf\_combined }\OtherTok{\textless{}{-}} \FunctionTok{get\_gfwData}\NormalTok{(}
  \AttributeTok{region =} \StringTok{"Australie"}\NormalTok{,}
  \AttributeTok{start\_date =} \StringTok{"2021{-}01{-}01"}\NormalTok{,}
  \AttributeTok{end\_date =} \StringTok{"2023{-}12{-}31"}\NormalTok{,}
  \AttributeTok{temp\_res =} \StringTok{"yearly"}\NormalTok{,}
  \AttributeTok{spat\_res =} \StringTok{"low"}\NormalTok{,}
  \AttributeTok{key =}\NormalTok{ key,}
  \AttributeTok{compress =} \ConstantTok{FALSE}
\NormalTok{)}

\NormalTok{data\_sf\_combined }\OtherTok{\textless{}{-}} \FunctionTok{get\_gfwData}\NormalTok{(}\StringTok{"Australie"}\NormalTok{,}
  \StringTok{"2019{-}01{-}01"}\NormalTok{,}
  \StringTok{"2023{-}12{-}31"}\NormalTok{,}
  \StringTok{"yearly"}\NormalTok{,}
  \AttributeTok{compress =} \ConstantTok{FALSE}
\NormalTok{)}
\end{Highlighting}
\end{Shaded}

\hypertarget{visualization}{%
\subsection{Visualization}\label{visualization}}

To display the data, we load : - The coastline from
\texttt{rnaturalearth} package and modify it to get an sf object, and we
constrain it to the boundaries of the given data. - EEZ Polygons from
\_Flanders Marine Institute (2023). Maritime Boundaries Geodatabase:
Maritime Boundaries and Exclusive Economic Zones (200NM), version 12.
Available online at \url{https://www.marineregions.org/}.
\url{https://doi.org/10.14284/632_}

\begin{Shaded}
\begin{Highlighting}[]
\CommentTok{\# Check and modify if necessary the spatial reference of data\_sf\_combined}
\NormalTok{data\_sf\_combined }\OtherTok{\textless{}{-}}\NormalTok{ sf}\SpecialCharTok{::}\FunctionTok{st\_set\_crs}\NormalTok{(}
\NormalTok{  data\_sf\_combined,}
\NormalTok{  sf}\SpecialCharTok{::}\FunctionTok{st\_crs}\NormalTok{(rnaturalearth}\SpecialCharTok{::}\FunctionTok{ne\_coastline}\NormalTok{(}\AttributeTok{scale =} \StringTok{"large"}\NormalTok{))}
\NormalTok{)}

\NormalTok{coast\_clipped }\OtherTok{\textless{}{-}}\NormalTok{ rnaturalearth}\SpecialCharTok{::}\FunctionTok{ne\_coastline}\NormalTok{(}\AttributeTok{scale =} \StringTok{"large"}\NormalTok{) }\SpecialCharTok{\%\textgreater{}\%}
\NormalTok{  sf}\SpecialCharTok{::}\FunctionTok{st\_as\_sf}\NormalTok{() }\SpecialCharTok{\%\textgreater{}\%}
\NormalTok{  sf}\SpecialCharTok{::}\FunctionTok{st\_intersection}\NormalTok{(sf}\SpecialCharTok{::}\FunctionTok{st\_as\_sfc}\NormalTok{(sf}\SpecialCharTok{::}\FunctionTok{st\_bbox}\NormalTok{(data\_sf\_combined)))}

\CommentTok{\# Load EEZ polygons}
\DocumentationTok{\#\# Downloaded data}
\CommentTok{\# eezs \textless{}{-} sf::st\_read(dsn = \textquotesingle{}\textasciitilde{}/2024\_Research\_A\_UQ/World\_EEZ\_v12\_20231025/\textquotesingle{}) \%\textgreater{}\%}
\CommentTok{\#   sf::st\_transform(crs = sf::st\_crs(data\_sf\_combined)) \%\textgreater{}\%}
\CommentTok{\#   sf::st\_make\_valid() \%\textgreater{}\%}
\CommentTok{\#   sf::st\_intersection(sf::st\_as\_sfc(sf::st\_bbox(data\_sf\_combined)))}

\NormalTok{eezs }\OtherTok{\textless{}{-}}\NormalTok{ mregions}\SpecialCharTok{::}\FunctionTok{mr\_shp}\NormalTok{(}\AttributeTok{key =} \StringTok{"Australia:eez"}\NormalTok{) }\SpecialCharTok{\%\textgreater{}\%}
\NormalTok{  dplyr}\SpecialCharTok{::}\FunctionTok{filter}\NormalTok{(geoname }\SpecialCharTok{==} \StringTok{"Australian Exclusive Economic Zone"}\NormalTok{) }\SpecialCharTok{\%\textgreater{}\%}
\NormalTok{  sf}\SpecialCharTok{::}\FunctionTok{st\_transform}\NormalTok{(}\AttributeTok{crs =}\NormalTok{ sf}\SpecialCharTok{::}\FunctionTok{st\_crs}\NormalTok{(data\_sf\_combined)) }\SpecialCharTok{\%\textgreater{}\%}
\NormalTok{  sf}\SpecialCharTok{::}\FunctionTok{st\_make\_valid}\NormalTok{() }\SpecialCharTok{\%\textgreater{}\%}
\NormalTok{  sf}\SpecialCharTok{::}\FunctionTok{st\_intersection}\NormalTok{(sf}\SpecialCharTok{::}\FunctionTok{st\_as\_sfc}\NormalTok{(sf}\SpecialCharTok{::}\FunctionTok{st\_bbox}\NormalTok{(data\_sf\_combined)))}
\end{Highlighting}
\end{Shaded}

\hypertarget{here-we-display-the-fishing-effort-in-australia-from-2019-to-2023.}{%
\paragraph{Here we display the Fishing Effort in Australia from 2019 to
2023.}\label{here-we-display-the-fishing-effort-in-australia-from-2019-to-2023.}}

\hypertarget{raw-fishing-effort}{%
\subparagraph{Raw Fishing Effort}\label{raw-fishing-effort}}

\includegraphics{gfwrPackageStraightforwardApproach_files/figure-latex/unnamed-chunk-10-1.pdf}

\hypertarget{by-years}{%
\subparagraph{By years}\label{by-years}}

\includegraphics{gfwrPackageStraightforwardApproach_files/figure-latex/unnamed-chunk-11-1.pdf}

\hypertarget{year-on-year-comparison}{%
\subparagraph{Year-on-year comparison}\label{year-on-year-comparison}}

We may need to compare different timeframes, such as seasons, to see if
there are any patterns. \textbf{Note :} As more vessels have adopted AIS
(mainly in economically developed countries) since the deployment of
these technologies, the rise in activities must be seen in the context
of this increase and not necessarily of more intense fishing activity.

\begin{Shaded}
\begin{Highlighting}[]
\CommentTok{\# We need to change the temporal range according to our need group by it to}
\CommentTok{\# display the total fishing hours. \textless{}br\textgreater{}}
\NormalTok{data\_sf\_combined }\OtherTok{\textless{}{-}} \FunctionTok{get\_gfwData}\NormalTok{(}
  \StringTok{"Australie"}\NormalTok{,}
  \StringTok{"2019{-}01{-}01"}\NormalTok{,}
  \StringTok{"2023{-}12{-}31"}\NormalTok{,}
  \StringTok{"monthly"}
\NormalTok{) }\SpecialCharTok{\%\textgreater{}\%}
\NormalTok{  dplyr}\SpecialCharTok{::}\FunctionTok{group\_by}\NormalTok{(Year, Month) }\SpecialCharTok{\%\textgreater{}\%}
\NormalTok{  dplyr}\SpecialCharTok{::}\FunctionTok{summarize}\NormalTok{(}\AttributeTok{Total\_Fishing\_Hours =} \FunctionTok{sum}\NormalTok{(}\StringTok{\textasciigrave{}}\AttributeTok{Apparent Fishing Hours}\StringTok{\textasciigrave{}}\NormalTok{))}
\end{Highlighting}
\end{Shaded}

\includegraphics{gfwrPackageStraightforwardApproach_files/figure-latex/unnamed-chunk-13-1.pdf}

\hypertarget{fishing-gear-type}{%
\subparagraph{Fishing gear type}\label{fishing-gear-type}}

Here we display the Vessel activity in `Micronesia' in 2022 according to
the fishing gear type.

\begin{Shaded}
\begin{Highlighting}[]
\NormalTok{data\_sf\_combined }\OtherTok{\textless{}{-}} \FunctionTok{get\_gfwData}\NormalTok{(}
  \StringTok{"Micronesia"}\NormalTok{,}
  \StringTok{"2022{-}01{-}01"}\NormalTok{,}
  \StringTok{"2022{-}12{-}31"}\NormalTok{,}
  \StringTok{"monthly"}\NormalTok{,}
  \StringTok{"low"}\NormalTok{,}
\NormalTok{  key}
\NormalTok{)}
\end{Highlighting}
\end{Shaded}

\includegraphics{gfwrPackageStraightforwardApproach_files/figure-latex/unnamed-chunk-16-1.pdf}

\hypertarget{flags}{%
\subparagraph{Flags}\label{flags}}

Here we display the Vessel activity in Papua New Guinea according to
Vessels flags.

\includegraphics{gfwrPackageStraightforwardApproach_files/figure-latex/unnamed-chunk-18-1.pdf}

\hypertarget{supplementary-materials.}{%
\subparagraph{Supplementary materials.}\label{supplementary-materials.}}

The fishing detection model was trained on AIS data from 503 vessels and
identified fishing activity with over 90\% accuracy, which means that it
can identify a fishing and non-fishing activity with high accuracy. More
details on AIS operation and limitations here :
\url{https://globalfishingwatch.org/dataset-and-code-fishing-effort/}

Hierarchy of vessels gear types :

\url{https://globalfishingwatch.org/datasets-and-code-vessel-identity/}*

\end{document}
